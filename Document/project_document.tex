
\documentclass[fleqn]{scrartcl}

\usepackage[utf8]{inputenc}
\usepackage[english]{babel}
\usepackage{amsmath}
\usepackage{amsfonts}
\usepackage{graphicx} 
\usepackage[hidelinks]{hyperref}
\usepackage{algorithm}
\usepackage[noend]{algpseudocode}





\title{Operations Research B}
\author{Lars Burghardt, Alexander Wördekemper, Ahmad Hashemi}
\date{22.02.2017}

\begin{document}
\maketitle
\tableofcontents
\section{Initial solution}

For our initial solution we decided to use a sequential construction algorithm with a hill climbing algorithm introduced in $[HoMu 12]$. We have chosen these algorithms because they seemed to produce good and fast initial solutions in the literature. We had to do some little adjustments because we could not open as many routes as we want because we have a fix number of vehicles. In line 1 we initialize an empty solution. Then we repeat the following until all customers have been inserted into a route or the number of routes is the number of vehicles we have: Initialize an empty route r (line 3). Then we have a for loop for all unassigned customers. In line 5 we get a random customer i. Then we insert the customer i at the end of the current route. In line 7 we call the hill climbing algorithm to improve the route r. If the new toot is feasible we mark customer i as inserted. If it is not feasible we remove i from the route again. After the for loop for all unassigned customers has finished we add the route r to the solution in line 12.
\\
\\  
In the hill climbing algorithm to improve the routes we have given a route r. Then we repeat the following until no improvement is achived in the previous pass: The for loop for each possible pair of locations starts (line 3). If the latter location is more urgent in its upper time window (line 4) then we swap the current two locations in r to get a new route r'. In line 6 we calculate the cost function for r and r'. If the new route has a better cost function value we set $r\gets r'$.

\begin{algorithm}
\caption{Sequential construction}\label{sequential}
\begin{algorithmic}[1]
\State Initialize an empty solution $s$
\Repeat
\State Initialize an empty route $r$
\For {(All unassigned customers)}
\State Get a random next customer $i$
\State Insert the customer $i$ at the end of the current route $r$
\State Call HC (Algorithm 2) to improve route $r$
\If {(route $r$ is feasible)}
\State Mark $i$ as inserted
\Else
\State Remove $i$ from route $r$
\EndIf
\EndFor
\State Add route $r$ to solution $s$
\Until {(All customers have been inserted OR $\vert s \vert = \vert Vehicles \vert$)}
\end{algorithmic}
\end{algorithm}

\begin{algorithm}
\caption{Hill climbing}\label{hillclimb}
\begin{algorithmic}[1]
\State Given a route $r$
\Repeat
\For {(Each possible pair of locations)}
\If {(The latter location is more urgent in its upper time window bound)}
\State Swap the current two locations in $r$ to get a new route $r'$
\State Calculate $cost(r)$ and $cost(r')$
\If {($cost(r)-cost(r') > 0$)}
\State $r \gets r'$
\EndIf
\EndIf
\EndFor
\Until {(Done)} \{Stop when no improvement achieved in the previous pass\}
\end{algorithmic}
\end{algorithm}


\subsection{Different approaches}
As we found out that we did not find initial solutions for larger instances we tried some different approaches to solve the problem

\subsubsection{Removing the customer with the largest distance}
The sequential construction algorithm adds random customers to the route and after the root gets improved by the hill climbing algorithm we remove the customer we had just added if the new route is infeasible. Instead of removing the just added customer we tried to remove the customer which needed the most time on the route. For every customer and it's destination we added the distance from the last node to the customer/destination and from the customer/destination to the next node. We removed the customer which needed the most time. In experiments we found out that this approach was no improvement compared to the one we had before.

\newpage
\section{Large neighborhood search}
\begin{algorithm}
\caption{LNS ($maxSize, range, iterations, probability$)}\label{lns}
\begin{algorithmic}[1]
\State Given a solution s
\State $current \gets s$
\For {($i \gets 2; i \leq masSize-range; i \gets i+1$)}
\For {($j \gets 0; j \leq range; j \gets j+1$)}
\For {($k \gets 0; k < iterations; k \gets k+1$)}
\State $new \gets$ Remove randomly $i+j$ customer in $current$
\For {(All removed customer)}
\State Add customer to a randomly selected route $r$ in $new$
\State Call HC (Algorithm 2) to improve route $r$
\EndFor
\State $pr \gets$ random number between 0 and 1
\If {($new$ is feasible solution)}
\If {($cost(new) < cost(current)$ OR $pr < probability$)}
\State $current = new$
\If {($cost(current) < cost(s)$)}
\State $s = current$
\EndIf
\EndIf
\EndIf
\EndFor
\EndFor
\EndFor
\end{algorithmic}
\end{algorithm}
\newpage
\section{Relevant work in the literature}
In the literature nearly all possible metaheuristics have been tested for the dial-a-ride problem. 
$[JaHe 11]$ introduced a large neighborhood search for the problem. It seemed to work well which is the reason why we decided to implement this metaheuristic. A different approach was introduced in $[CoLa 03]$ where they used a tabu search heuristic for the dial-a-ride problem. In $[Parragh$  $et$  $al.$ $10]$ they used a variable neighborhood search to solve the problem. In $[Jørgensen$ $et$ $al.$ $07]$ genetic algorithms were introduced the solve the dial-a-ride problem.
Every approach has its pros and cons but in most cases the large neighborhood search gets the best solutions in the literature.
\newpage
\section{How to compile and run the code}
Our approach is implemented in C\# using Microsoft Visual Studio Enterprise 2015 under Windows. To compile and run the code, simply open the solution file \textit{ORB.DARP.sln} and run the project using \textit{F5} (debug) or \textit{Shift + F5} (no debug).
\\
A copy of Microsoft Visual Studio Enterprise 2015 can be downloaded at \href{https://dreamspark.uni-paderborn.de/}{Dreamspark UPB}.
\\
\\
Another method to run the code, is to open a command line window and enter the path to the compiled \textit{orbdar.exe} with the additional parameters.
\newpage

\section{Experimental investigation of our approach's components}
ToDo
\newpage

\section{Experimental investigation of our approach's performance}
ToDo
\newpage


\section{Literature}
$[HoMu 12]$ M. I. Hosny and C. L.Mumford, “Constructing initial solutions for the multiple vehicle pickup and delivery problem with time windows”, Journal of King Saud University, Computer and Information Sciences, vol. 24, no. 1, pp. 59–69, 2012.
\\
\\
$[JaHe 11]$ S. Jain , P. Van Hentenryck, “Large neighborhood search for dial-a-ride problems”, In: Principles and practice of constraint programming, Notes in computer science, vol. 6876. Springer, 2011.
\\
\\
$[CoLa 03]$ J.-F. Cordeau, G. Laporte, "A tabu search heuristic for the static multi- vehicle dial-a-ride problem", Transportation Research Part B 37: 579–594, 2003
\\
\\
$[Parragh$  $et$  $al.$ $10]$ S.N. Parragh, K.F. Doerner, R.F. Hartl, "Variable neighborhood search for the dial-a-ride problem", Computers \& Operations Research 37 (6),1129–1138, 2010. 
\\
\\
$[Jørgensen$ $et$ $al.$ $07]$ R.M. Jørgensen , J. Larsen, K.B. Bergvinsdottir, "Solving the dial-a-ride problem using genetic algorithms", J Oper Res Soc 58:1321–1331, 2007.
\end{document}